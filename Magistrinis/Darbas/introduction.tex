
Objekto erdvėje koordinačių suradimas yra plačiai pritaikomas uždavinys.
Kasdieniniame gyvenime žmonėms padeda orientuotis jiems nepražystamose vietovėse ir greičiau nuvykti į norimą vietą.
Vietos nustatymo uždavinys pritaikomas ir versle. 
Logistikos bendrovės gali nuolat stebėti savo krovinių siuntas, kas padeda greičiau paskirstyti srautus ir operatyviau spręsti apie esamus rinkos poreikius.
Ypatingą reikšmę lokacija turi karo pramonėje, kur laivų, lėktuvų ir kitų karinių priemonių stebėjimas reikalingas, nepaisant sudėtingų sąlygų.
Todėl šis uždavinys yra aktualus kiekvienam.

\iffalse 
moksle (?)
\fi

Ryšio operatoriai spendžia sudėtingą uždavinį modeliuodami bazinių stočių išdėstymą, kadangi nustatyti vartotojo vietą, panaudojant trianguliacines sistemas, reikia užmegzti ryšį bent su trimis paieškos siųstuvais. O radus optimalų išdėstymą galima sumažinti energijos sunaudojimo kaštus, pagerinti signalo kokybę, sumažinti interferenciją, padidinti apimamą veiklos plotą ir palengvinti sistemos įrengimą bei eksploataciją.

Šio darbo tikslas yra susipažinti su algoritmais padėsiančiais modeliuoti, trianguliacija naudojančias, lokacines sistemas bei rasti optimalų siųstuvų išdėstymą.    

