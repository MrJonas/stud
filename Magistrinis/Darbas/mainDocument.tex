% Kompiuterijos katedros šablonas
% Template of Department of Computer Science II
% Versija 1.0 2015 m. kovas [ March, 2015]

\documentclass[a4paper,12pt,fleqn]{article}
\input{allPacks}

\newtoggle{inLithuanian}
 %If the report is in Lithuanian, it is set to true; otherwise, change to false
\settoggle{inLithuanian}{true}

%create file preface.tex for the preface text
%if preface is needed set to true
\newtoggle{needPreface}
\settoggle{needPreface}{false}

\newtoggle{signaturesOnTitlePage}
\settoggle{signaturesOnTitlePage}{true}


\input{macros}

\begin{document}
 % #1 -report type, #2 - title, #3-7 students, #8 - supervisor
 \depttitlepage{Mokslo tiriamasis darbas I}{Statinių trianguliacijos siųstuvų lokacijos optimizacija}{Jonas Antanaitis} 
 {}{}{}{}% students 2-5
 {dr. Valdas Rapševičius}

\tableofcontents

\iffalse
%keywords and notations if needed
\sectionWithoutNumber{Sutartinis terminų žodynas}{keywords}{Pateikiamas terminų sąrašas (jei reikia)}
\fi

 %both abstracts
\bothabstracts{\input{abstract}}%tex-file of abstract in original language
{Transmitter Anchor Locations Optimization for Triangulation} %if work is in LT this title should be in English
{\input{abstractEN}}%tex-file of abstract in other language


 %Introduction section: label is sec:intro
\sectionWithoutNumber{\keyWordIntroduction}{intro}

Objekto erdvėje koordinačių suradimas yra plačiai pritaikomas uždavinys.
Kasdieniniame gyvenime žmonėms padeda orientuotis jiems nepražystamose vietovėse ir greičiau nuvykti į norimą vietą.
Vietos nustatymo uždavinys pritaikomas ir versle. 
Logistikos bendrovės gali nuolat stebėti savo krovinių siuntas, kas padeda greičiau paskirstyti srautus ir operatyviau spręsti apie esamus rinkos poreikius.
Ypatingą reikšmę lokacija turi karo pramonėje, kur laivų, lėktuvų ir kitų karinių priemonių stebėjimas reikalingas, nepaisant sudėtingų sąlygų.
Todėl šis uždavinys yra aktualus kiekvienam.

\iffalse 
moksle (?)
\fi

Ryšio operatoriai spendžia sudėtingą uždavinį modeliuodami bazinių stočių išdėstymą, kadangi nustatyti vartotojo vietą, panaudojant trianguliacines sistemas, reikia užmegzti ryšį bent su trimis paieškos siųstuvais. O radus optimalų išdėstymą galima sumažinti energijos sunaudojimo kaštus, pagerinti signalo kokybę, sumažinti interferenciją, padidinti apimamą veiklos plotą ir palengvinti sistemos įrengimą bei eksploataciją.

Šio darbo tikslas yra susipažinti su algoritmais padėsiančiais modeliuoti, trianguliacija naudojančias, lokacines sistemas bei rasti optimalų siųstuvų išdėstymą.    





 %the main part
\newpage
\section{Lokacijos sistemos}
\label{sec:motivation}
\subsection{Sklidimo modelis}
\subsection{Apytikslės lokacijos sistemos}
\label{sec:example}
Pateikiamas \ref{sec:example} poskyrio tekstas.

Vienas iš šaltinių~\cite{L1}.

Vienas iš šaltinių~\cite{L2}.

Vienas iš šaltinių~\cite{L3}.

Vienas iš šaltinių~\cite{L4}.
 
Vienas iš šaltinių~\cite{L5}.

Visas turinys priklauso \ref{sec:motivation} skyriui.

\subsection{Trianguliacinės lokacijos sistemos}
\subsubsection{Pagal kampą}
\subsubsection{Pagal signalo stiprumą}
\subsubsection{Pagal signalo sklidimo laiką}



\section{Trianguliacijos siųstuvų lokacijos optimizacijos algoritmai}
\subsubsection{DIRECT}
\subsubsection{Kontroliuojama sankirta}
\subsubsection{kt}


\section{Genetiniai algoritmai}

\iffalse
\begin{equation}
x = \sum_{i=1}^N m_i
\end{equation}

\begin{table}[!ht]\centering
\caption{Lentelė ... }
\label{tabl:table}
\begin{tabular}{l|r|}
test&test\\ \hline
test&test\\
\end{tabular}
\end{table}
\fi

 %Conclusions section
\sectionWithoutNumber{\keyWordConclusions}{conclu}
\input{conclusions.tex}

%ateities darbų gairės, planas/next steps of the work
\sectionWithoutNumber{Ateities tyrimų planas}{future}{Pristatomi ateities darbai ir/ar jų planas, gairės tolimesniems darbams....}

 %file literatureSources.bib
\referenceSources{literatureSources}


\iffalse
%% this part is optional
\newpage
\begin{appendices}
Dokumentą sudaro du priedai: \ref{app:a} priede  ....
\newpage
\section{Pirmojo priedo pavadinimas}
\label{app:a}
Pirmojo priedo tekstas ...

\newpage
\section{Antrojo priedo pavadinimas}
Antrojo priedo tekstas ...

\end{appendices}
\fi

\end{document}
